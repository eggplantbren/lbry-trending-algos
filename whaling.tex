\documentclass[a4paper, 12pt]{article}
\usepackage{amsmath}
\usepackage{amssymb}
\usepackage{dsfont}
\usepackage[left=2cm, right=2cm, bottom=3cm, top=2cm]{geometry}
\usepackage{graphicx}
\usepackage[utf8]{inputenc}
\usepackage{microtype}
\usepackage{natbib}

\title{Whaling}
\author{Brendon J. Brewer}
\date{}

\begin{document}
\maketitle

%\abstract{\noindent Abstract}

% Need this after the abstract
\setlength{\parindent}{0pt}
\setlength{\parskip}{8pt}

\section{AR setup}
Let $x$ be an old total amount and $x'$ be the new one. The updated trending
score $y'$ is then
\begin{align}
y' &= ky + f(x, x', y)
\end{align}
where $y$ is the old trending score, $k$ is the decay coefficient, and
$f(x, x', y)$ is the spike height function.

Suppose a whale has $L$ LBC and shifts it onto a claim with initial amount
close to zero and initial trending score close to zero.
The trending score will jump to
\begin{align}
y' &= k \times 0 + f(L, 0, 0)
\end{align}
Under the current {\tt ar.py}, this is (ignoring the minnow boost)
\begin{align}
y' \approx L^{1/4}.
\end{align}

If this happens at time $t=0$ (in units of blocks), it will decay according to
\begin{align}
y(t) &= L^{1/4} k^t. \label{eqn:decay}
\end{align}
The half life of the decay is $\ell = -1/\log_2(k)$. If we assume that a whale
does this every $\ell$ blocks, and that the probability distribution
over when he started to do it is uniform, then we can obtain the
stationary distribution over trending score:
Equation~\ref{eqn:decay} will be its inverse CDF.
\begin{align}
F^{-1}(u) = L^{1/4} k^{\ell u}.
\end{align}
Inverting this gives the CDF
\begin{align}
F(y) &= \frac{\ln y - \frac{1}{4}\ln L}{\ell \ln k}.
\end{align}

Differentiating gives the PDF:
\begin{align}
f(y) &= \frac{1}{\ell \ln k} \frac{1}{y},
\end{align}
Where $y \geq L/2$.




\bibliographystyle{plainnat}
\bibliography{references}

\end{document}

